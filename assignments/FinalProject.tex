\documentclass[10pt]{article}

\usepackage[margin=1in]{geometry}
\usepackage{amsmath,amsthm,amssymb}
\usepackage{graphicx}
\usepackage{subfig}
\usepackage{float}
\usepackage{wrapfig}
\usepackage{multicol}
\usepackage{booktabs}
\usepackage{hyperref}
\usepackage{pythonhighlight}

 \hypersetup{
 	colorlinks=true,
 	linkcolor=blue,
 	filecolor=magenta,      
 	urlcolor=cyan,
 }

\setlength\parindent{0pt}

\graphicspath{ {/Users/clay/Documents/research/TGE-SP21/assignments/images/} }

\newcommand{\titler}[1]{
	\title{Final Project} %replace X with the appropriate number
	\author{Geosc 597-003 \\
		Techniques of Geophysical Experimentation} %if necessary, replace with your course title \\
	\date{Due: #1}
	
	\maketitle}


\begin{document}

% --------------------------------------------------------------
%                         Start here
% --------------------------------------------------------------

\titler{07 May}

\section*{Final Project}
Each student will give a 15 minute presentation on their final project. You should be able to demonstrate the functionality of your \textit{working} prototype in class -- this can be in-person or via media, depending on the appropriateness. 

\subsection*{Details:}
Cover the following in your presentation...
\begin{itemize}
	\item Problem statement -- give necessary context/motivation so that a broad audience can appreciate.
	\item What is the proposed solution to the problem you've identified?
	\item Demonstrate your working prototype and describe the functionality.
	\begin{itemize}
		\item Explain how it works (mechanics, electronics, code, etc.) Use diagrams or photos to help explain.
		\item What were the design considerations?
		\item Describe any unforeseen obstacles, considerations, limitations that impacted the implementation of your prototype.
	\end{itemize}
	\item Describe in detail what changes/alterations/improvements you would make. How would the next version be different?
\end{itemize}


\end{document}
