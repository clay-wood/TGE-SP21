\documentclass[10pt]{article}

\usepackage[margin=1in]{geometry}
\usepackage{amsmath,amsthm,amssymb}
\usepackage{graphicx}
\usepackage{subfig}
\usepackage{float}
\usepackage{wrapfig}
\usepackage{multicol}
 \usepackage{booktabs}
 \usepackage{hyperref}
 \hypersetup{
 	colorlinks=true,
 	linkcolor=blue,
 	filecolor=magenta,      
 	urlcolor=cyan,
 }

\setlength\parindent{0pt}

\graphicspath{ {/Users/clay/Documents/research/TGE-SP21/assignments/images/} }

\newcommand{\titler}[2]{
	\title{Assignment #1} %replace X with the appropriate number
	\author{Geosc 597-003 \\
		Techniques of Geophysical Experimentation} %if necessary, replace with your course title \\
	\date{Due: #2}
	
	\maketitle}


\begin{document}

% --------------------------------------------------------------
%                         Start here
% --------------------------------------------------------------

\titler{5}{29 March}

\section*{Activity: Pylook Data Reduction}

\begin{wraptable}{r}{5cm}
	%	\begin{table}[h!t]
	\footnotesize
	\centering
	\begin{tabular}{@{}ll@{}}
		\multicolumn{2}{c}{\textbf{Grading Rubric}} \\ \midrule 
		\multicolumn{1}{l}{\textit{Topic}}   & \textit{Points}   \\ \midrule 
		jupyter notebook   & 70   \\ \midrule
		brief report w/ plots   & 15   \\ \midrule
		reduced data file & 5  \\ \midrule
		README & 5  \\ \bottomrule
	\end{tabular}
	%	\end{table}
\end{wraptable}

Use pylook to reduce the raw data from an biax friction experiment. Then generate a brief report with plots, a processed data file, and a Readme file. 

\begin{itemize}
	\item Load in \textit{p5468S04Mug5l} -- the `l-file'.
	\item Note: data['Time'] = np.cumsum(data['Time']/10000) -- b/c we down-sample from 10kHz recording rate. 
	\item Convert from bit to physical units.
	\begin{itemize}
		\item The NI recorder is a 24-bit system
		\item Calibrations are found on the experiment runsheet
	\end{itemize}
	\item Correct for any offsets (review notes of runsheet for reference).
	\item Correct for zeros points in data. 
	\item Calculate the coefficient of friction. Is this reasonable -- why/not? Cite source(s).
	\item Include plots in the Jupyter Notebook.
\end{itemize}

\section*{What to upload to Canvas}
You should upload the following with \textit{consistent file names}:

\begin{itemize}
	\item Jupyter Notebook
	\begin{itemize}
		\item \textbf{Fully comment your code!} 
		\item \textbf{Naming convention:} p5468\_look\_username.ipynb
		\item Put all Arduino codes for this assignment in a project folder. 
	\end{itemize}
	\item Brief report
	\begin{itemize}
		\item Brief description of experiment.
		\item Relevant figures with proper labels and brief captions (one figure per page).
	\end{itemize}
	\item Reduced/processed data file in a standard format (e.g. *.csv, *.txt). 
	\item README -- a file that describes files in directory. 
	\item Zip directory and upload to Canvas. (p5468\_username.zip)
\end{itemize}


\end{document}
