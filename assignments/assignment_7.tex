\documentclass[10pt]{article}

\usepackage[margin=1in]{geometry}
\usepackage{amsmath,amsthm,amssymb}
\usepackage{graphicx}
\usepackage{subfig}
\usepackage{float}
\usepackage{wrapfig}
\usepackage{multicol}
\usepackage{booktabs}
\usepackage{hyperref}
\usepackage{pythonhighlight}

 \hypersetup{
 	colorlinks=true,
 	linkcolor=blue,
 	filecolor=magenta,      
 	urlcolor=cyan,
 }

\setlength\parindent{0pt}

\graphicspath{ {/Users/clay/Documents/research/TGE-SP21/assignments/images/} }

\newcommand{\titler}[2]{
	\title{Assignment #1} %replace X with the appropriate number
	\author{Geosc 597-003 \\
		Techniques of Geophysical Experimentation} %if necessary, replace with your course title \\
	\date{Due: #2}
	
	\maketitle}


\begin{document}

% --------------------------------------------------------------
%                         Start here
% --------------------------------------------------------------

\titler{7}{23 April}

\section*{Analog to Digital Conversion \& Quantization}

Read section 1 of the document \textit{From analog to digital} -- subsections 1 - 4 should be informative. Use the python notebook\textit{ quantization\_animated.ipynb} to interactively explore signal quantization. These files are available under the \textit{resources} directory in the Git repository. \\ \\
There are 4 cells in quantization\_animated.ipynb. \\
\textbf{Cell [1]} imports all necessary libraries -- you may need to install some  via anaconda prompt or terminal. \\
\textbf{Cell [2]} defines a function that creates a sinusoidal waveform by solving the wave equation.\\
\textbf{Cell [3]} creates an interactive plot with analog and quantized signals. How does increasing the bit depth affect the quantized signal? \\
\textbf{Cell [4]} \textcolor{red}{is broken -- you need to make it work.} Calculate the error between the signal and the quantized signal and the signal to noise ratio (SNR).
\begin{python}
# CALCULATE ERROR BETWEEN SAMPLED SIGNAL AND QUANTIZED SIGNAL
error = 

# CALCULATE SIGNAL TO NOISE RATIO (SNR)
# FIRST: CALCULATE MEAN POWER OF INPUT SIGNAL -- Ps
# SECOND: CALCULATE MEAN POWER OR VARIANCE OF QUANTIZATION RROR -- Pe
# SNR = Ps/Pe

Ps = 
Pe = 
SNR = 
\end{python}


\section*{Final Project -- Proposal}
The project should help you solve a research problem you are facing while applying some of the skills covered in this course. We would like to help you define a project that is within the scope of the course and the semester time limit on the work. A good project will have several characteristics:

\begin{itemize}
	\item Uses multiple disciplines we discuss (i.e. mechanical, electrical, software, etc.)
	\item Solves a problem with no commercial solution or no economical commercial solution. 
	\item Will not cost thousands of dollars to build. Plan on \$50 or less.
	\item Does not put you or others in the path of potential harm or danger.
\end{itemize}

Write a summary of one page or less that describes:
\begin{itemize}
	\item What your project is.
	\item What it will do.
	\item What resources you will need to complete it.
	\item What parts you may need and a rough estimate of the cost.
\end{itemize}

\section*{What to upload to Canvas}
You should upload the following with \textit{consistent file names}:

\begin{itemize}
	\item Working python notebook -- \textbf{50 pts}
	\begin{itemize}
		\item \textbf{Fully comment your code!} 
	\end{itemize}
	\item Final project proposal -- \textbf{50 pts}
	\item Zip directory and upload to Canvas. (hw7\_username.zip)
\end{itemize}

\end{document}
